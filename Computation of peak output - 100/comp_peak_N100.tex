\documentclass[a4paper,11pt]{article}
\usepackage{amsmath} % for math mode
\usepackage{amssymb} % for math symbol
\usepackage{graphicx} % for including graphics
\usepackage{indentfirst} % for indenting the first paragraph of a section





% for theorem mode
\usepackage{amsthm}
\newtheorem{thm}{Theorem}
\newtheorem{prop}{Proposition}
\newtheorem{asmp}{Assumption}
\newtheorem{lem}{Lemma}
\newtheorem{rmk}{Remark}
\theoremstyle{definition}
\newtheorem{dfn}{Definition}


%% paper layout
\pagestyle{headings}% page numbering style
%\setlength{\parskip}{5mm} % gab between two paragraphs
\linespread{1.3} % gap between tow lines
% adjust horizontal layout
\setlength{\hoffset}{-20pt}
\setlength{\textwidth}{440pt}  
% adjust vertical layout
\setlength{\voffset}{-40pt}
\setlength{\textheight}{700pt}





%>>>>>>>>>>>>>>>>>>>>>>>>>>>>>>>>>>>>><<<<<<<<<<<<<<<<<<<<<<<<<<<<<<<<<<
%--------------------------------------------------------------------- begin document ----------------------------------------------------
%>>>>>>>>>>>>>>>>>>>>>>>>>>>>>>>>>>>>><<<<<<<<<<<<<<<<<<<<<<<<<<<<<<<<<<
\begin{document}
\begin{center}
{\huge \bf Computation of Peak Output for Persistent Inputs} 
\end{center}
\section{Computation of peak output}
Consider a linear time-invariant and non-anticipative system whose input  $f:\mathbb{R}\rightarrow\mathbb{R}$ 
    and output $y:\in\mathbb{R}\rightarrow\mathbb{R}$ are related by
\begin{equation}\label{eq:sys}
    y(t,f) = \int_{-\infty}^\infty h(t-\tau)f(\tau)d\tau
\end{equation}
where $h:\mathbb{R}\rightarrow\mathbb{R}$ is the impulse response of the system.
Suppose that $h$ takes the form
\[
    h(t) = \beta\delta(t) + \bar{h}(t)
\]
where $\beta\in\mathbb{R}$, $\delta$ denotes the Dirac delta function and 
    $\bar{h} :\mathbb{R}\rightarrow\mathbb{R}$ 
    is a bounded and piecewise continuous function.
    

Suppose that $f$ belong to the set $\mathcal{F}$ characterized by
\begin{equation}\label{eq:F}
    \mathcal{F}\triangleq \{f\in\mathbb{L}_\infty: \|f\|_\infty\le M\;\; \text{and}\;\; \|\dot{f}\|\le D\}
\end{equation}
    where $M$ and $D$ are positive numbers.
With the input set $\mathcal{F}$, define the peak output of $y$ by
\begin{equation}\label{peaky1}
    \hat{y}\triangleq \sup_{f\in\mathcal{F}}\|y(f)\|_\infty.
\end{equation}
The peak output $\hat{y}$ can be determined by considering just one particular time $t$. 
Consequently, it is convenient to consider the case in which the peak output occurs at $t=0$.
Hence, the problem \eqref{peaky1} becomes
\begin{equation}\label{peaky2}
    \hat{y} = \sup_{f\in\mathcal{F}} J(f) 
\end{equation}
where $J(f)$ is the cost function given by
\begin{equation}\label{eq:int}
    J(f)\triangleq \beta f(0) + \int_{-\infty}^\infty \bar{h}(-\tau)f(\tau)d\tau.
\end{equation}



Now the $\infty$-norms of $f$ and $\dot{f}$ are considered for $t\in[-T,T]$ instead of $t\in(-\infty,\infty)$.
To this end, the trajectories of $\bar{h}(t)$ and $f(t)$ for $t\in[-T,T]$ are represented by vectors 
    $\mathbf{h}$ and $\mathbf{f}$ such that
\[
   \mathbf{h} \triangleq [h_{-n},h_{-n+1},\ldots,h_{n-1},h_n]^T\in\mathbb{R}^{2n+1}\quad \text{and}\quad 
  \mathbf{f}_0 \triangleq [f_{-n},f_{-n+1},\ldots,f_{n-1},f_n]^T \in\mathbb{R}^{2n+1}
\]
where $h_i = h(t_i)$ and $f_i = f(t_i)$.
The time point $t_i$ is given by
\[
    t_{-n} = -T,\quad t(i+1) = t_i+\sigma,\;\; i = -n,\-n+1,\ldots,n-1.
\]
Following Lane (1992), $\mathbf{f}_0$ is replaced by
 \[
       \mathbf{f} \triangleq [f_{-n+1},f_{-n+2},\ldots,f_{n-1},f_n]^T \in\mathbb{R}^{2n}.
 \]
 By employing the Simpson’s rule, the cost function $J(f)$ can be approximated by
\begin{equation}\label{eq:cost_new}
    J(f) \approx \mathbf{c}^T\mathbf{f}
\end{equation}
    where $\mathbf{c}$ is given by
\begin{equation}\label{eq:c}
    \mathbf{c}\triangleq \frac{\sigma}{3}[4h_{n-1},2h_{n-2},4h_{n-3},\ldots,2h_2,4h_1,\frac{3\beta}{\sigma}+h_0,
                                      \mathbf{0}_{1\times n}]^T\in\mathbb{R}^{2n}.
\end{equation}


Consider the constraints on the input $f$ in \eqref{eq:F}.
The constraint $\|f\|_\infty\le M$ is replaced by
\begin{equation}\label{eq:constrM}
    I_{2n\times 2n}\mathbf{f} \preceq M\mathbf{1}_{2n\times 1},\quad\text{and}\quad  
    -I_{2n\times 2n}\mathbf{f} \preceq M\mathbf{1}_{2n\times 1}.    
\end{equation}
By using the first-order backward formula, the constraint $\|\dot{f}\|_\infty\le D$ is replaced by
\begin{equation}\label{eq:constrD}
    Q_d\mathbf{f} \preceq D\mathbf{1}_{2n\times 1},\quad\text{and}\quad 
    -Q_d\mathbf{f} \preceq D\mathbf{1}_{2n\times 1}
\end{equation}
where 
\[
Q_d =
\begin{bmatrix}
 1 & 0 & 0 & \ldots & 0 & 0 & 0 \\
-1 & 1 & 0 & \ldots & 0 & 0 & 0  \\
0 & -1 & 1 & \ldots & 0 & 0 & 0  \\
\vdots & \vdots & \vdots &\ddots & \vdots & \vdots & \vdots  \\
 0 & 0 & 0 & \ldots & -1 & 1 & 0  \\
 0 & 0 & 0 & \ldots & 0 & -1 & 1  
\end{bmatrix}\in\mathbb{R}^{2n\times 2n}.
\]


By using \eqref{eq:cost_new}--\eqref{eq:constrD},
    the approximated peak output is the solution of the following optimisation problem:
\begin{equation}\label{eq:LP}
    \begin{array}{cc}
    \displaystyle{\max_{\mathbf{f}\in\mathbb{R}^{2n}}} & \mathbf{c}^T\mathbf{f}\\
    \text{subject to} & A\mathbf{f} \preceq \mathbf{b}
    \end{array}
\end{equation}    
where 
\[
    A = \begin{bmatrix} I_{2n\times 2n}\\ -I_{2n\times 2n}\\ Q_d\\ -Q_d\end{bmatrix},\quad 
    \mathbf{b} =  \begin{bmatrix} M\mathbf{1}_{2n\times 1}\\ M\mathbf{1}_{2n\times 1}\\
                                                   D\mathbf{1}_{2n\times 1}\\ D\mathbf{1}_{2n\times 1}\end{bmatrix}
\]
    and $\mathbf{c}$ is given in \eqref{eq:c}.
    
    
\newpage
% ------------------------------------------------------------------------------------------------------------
\section{Numerical example}
Consider the system described by
\[
    H(s) = \frac{\omega_n^2}{s^2 + 2\zeta\omega_n s + \omega_n^2}
\]
where 
\[
    \omega_n = 10 \ \mathrm{rad/sec} \quad \text{and}\quad \zeta = 0.8.
\]
Suppose that the system input belongs to the set $\mathcal{F}$ in \eqref{eq:F} with
\[
    M = 1\quad \text{and}\quad D = 5.
\]
The number of sampling intervals $N$ and the truncation time $T$ are selected to be
\[
    N = 100\quad \text{and}\quad T = 2.5.
\]


By using Sedumi package in Matlab, we obtain the peak output of the system as shown in the table.



%------------------------------------------------------------ตาราง
\begin{table}[h]
\begin{center}
\vspace{1cm}
\begin{tabular}{|c|c|c|c|}
\hline
 exact value  & peak output & error (\%)& CPU Time (s)\\ 
\hline
$1.180$ & $1.008$ & $14.58\%$ & $0.0267$\\
\hline
\end{tabular}
\end{center}
\end{table}

\noindent
Computer spec.: CPU Intel(R) Core(TM) i7 2.30 GHz RAM 8 GB
%----------------------------------------------------------กราฟ
\begin{figure}[h]
\begin{center}
\includegraphics[scale=0.8]{max_inp.eps}
\end{center}
\caption{Maximal input}
\end{figure}
\end{document}
%>>>>>>>>>>>>>>>>>>>>>>>>>>>>>>>>>>>>><<<<<<<<<<<<<<<<<<<<<<<<<<<<<<<<<<
%----------------------------------------------------------------------- end document ----------------------------------------------------
%>>>>>>>>>>>>>>>>>>>>>>>>>>>>>>>>>>>>><<<<<<<<<<<<<<<<<<<<<<<<<<<<<<<<<<





